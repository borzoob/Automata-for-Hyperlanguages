
\begin{abstract}

{\em Hyperproperties} lift conventional trace properties from a set of 
execution traces to a set of sets of execution traces. Hyperproperties have 
been shown to be a powerful formalism for expressing and reasoning about 
information-flow security policies and important properties of cyber-physical 
systems such as sensitivity and robustness as well as consistency conditions in 
distributed computing such as linearizability. Although there is an extensive 
body of work on automata-based representation of trace properties, we currently 
lack such characterization for hyperproperties.

In this paper, we first propose an automata representation for 
finite hyperlanguages, called {\em nondeterministic finite 
automaton over hyperwords} (NFH). These automata allow running multiple 
quantified traces. Then, we explore the fundamental properties of NFH and show 
closure under complementation, union, and intersection. We also show that 
in general, while the membership problem is decidable the emptiness problem is 
undecidable. Finally, we study learning NFH. We introduce learning algorithms 
inspired by Angluin’s L* algorithm for the fragments NFH, where the trace 
quantification is either universal or existential.

\end{abstract}