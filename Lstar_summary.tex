\subsubsection{The $\lstar$ algorithm}

$\lstar$ consists of two entities: a {\em learner}, who wishes to learn a DFA $A$ for an unknown (regular) language $\cal L$, and a {\em teacher}, who knows $\cal L$.
During the learning process, the learner asks the teacher two types of queries: {\em membership queries} (``is the word $w$ in $\cal L$?'') and {\em equivalence queries} (``Is $A$ a DFA for $\cal L$?'').

The learner maintains $A$ in the form of an {\em observation table} $T$ of truth values, whose rows $D, D\cdot\Sigma$ and columns $E$ are sets of words over $\Sigma$, where $D$ is prefix-closed, and $E$ is suffix-closed. Initially, $D = E = {\epsilon}$. 
For a row $d$ and a column $e$, the entry for $T(d,e)$ is $\true$ iff $d\cdot e \in{\cal L}$. The entries are filled via membership queries.
The vector of truth values for row $d$ is denoted $\row(d)$. Intuitively, the rows in $D$ determine the states of $A$, and the rows in $D\times\Sigma$ determine the transitions of $A$: the state $\row(d\cdot\sigma)$ is reached from $\row(d)$ upon reading $\sigma$. 

The learner updates the table until it is both {\em closed} and {\em consistent}. The table is closed if for every $d\in D$ and $\sigma\in \Sigma$, there is $d'\in D$ such that $\row(d') = \row(d\cdot\sigma)$. Intuitively, a closed table assures a full transition relation.
The table is consistent if for every $d_1,d_2\in D$ and every $\sigma \in \Sigma$, it holds that if $\row(d_1) = \row(d_2)$, then $\row(d_1\cdot\sigma) = \row(d_2\cdot\sigma)$. Intuitively, a consistent table assures a deterministic transition relation. 

If the table is not closed, then the missing row $d'$ is added to $D$, and $d'\cdot\sigma$ is added to $D\times\Sigma$ for every $\sigma\in \Sigma$. The new entries in $T$ are filled via membership queries. Note that this leaves $D$ prefix-closed.
If the table is inconsistent, then there is $e\in E$ for which $d_1\cdot\sigma\cdot e \neq d_2\cdot\sigma\cdot e$. The word $\sigma\cdot e$ then separates $\row(d_1)$ from $\row(d_2)$, and is added to $E$ (notice that this leaves $E$ suffix-closed). The new table entries are filled accordingly, and now $\row(d_1)\neq \row(d_2)$. 

When $T$ is closed and consistent, the learner constructs $A$: The states are the rows of $D$, the initial state is $\row(\epsilon)$, the accepting states are these in which $T(d,\epsilon) = \true$, and the transition relation is as described above. The learner then submits an equivalence query to the teacher. The teacher either confirms the equivalence, in which case the algorithm terminates, or returns a counterexample $w\in \lang{A}$ but $w\notin {\cal L}$ (which we call a {\em positive counterexample}), or $w\notin \lang{A}$ but $w\in {\cal L}$ (which we call a {\em negative counterexample}). The learner then adds $w$ and all its suffixes to $E$, and proceeds to construct the next candidate DFA $A$. 

In is shown in \cite{Angluin87} that as long as $A$ is not a DFA for the target language $\cal L$, it has less states than a minimal DFA for $\cal L$. Further, every change in the table adds at least one state to $A$. Therefore, the procedure is guaranteed to terminate successfully with a minimal DFA $A$ for $\cal L$. 