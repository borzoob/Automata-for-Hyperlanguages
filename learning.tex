\section{Learning NFH}

In this section, we introduce learning algorithms for the fragments $\nfhf$ and $\nfhe$. Our algorithms are based on Angluin's $\lstar$ algorithm \cite{Angluin87} for regular languages.

We first survey the $\lstar$ algorithm, and then describe the relevant adjustments for learning $\nfhe$ and $\nfhf$.

\input{Lstar_summary}

The correctness of the $\lstar$ algorithm follows from the fact that regular languages have a {\em canonical form}, which guarantees a single minimal DFA for a regular language $L$. To enable an $\lstar$-based algorithm for  $\nfhf$ and $\nfhe$, we first define canonical forms for these fragments. 

\input{canonical_forms}


\subsection{Learning $\nfhf$ and $\nfhe$}

We now describe learning algorithms for $\nfhe$ and $\nfhf$, based on the $\lstar$ framework.
These algorithms aim to learn minimal deterministic sequence-complete (in the case of $\nfhf$) or permutation-complete (in the case of $\nfhe$) NFH, with a minimal number of variables, for a target hyperlanguage $\hl$.

In the case of hyperautomata, the membership queries and the counterexamples provided by the teacher consist of hyperwords. We assume that the teacher returns a minimal counterexample in terms of size of the hyperword. 

During the procedure, the learner maintains an NFH $\A$ via the current number of variables $k$ and an observation table for $\hat\A$, over the alphabet $\hat\Sigma  = (\Sigma\cup\{\#\})^k$, where $k$ is initially set to $1$. 
When the number of variables is increased to $k'>k$, the alphabet of $\hat\A$ is extended accordingly to $(\Sigma\cup\{\#\})^{k'}$. 
To this end, we define a function $\uparrow_k^{k'}:(\Sigma\cup\{\#\})^k\rightarrow (\Sigma\cup\{\#\})^{k'}$, which replaces every word $(\sigma_1,\ldots \sigma_k)$, with $(\sigma_1,\ldots \sigma_k,\ldots \sigma_k)$. That is, the last letter is duplicated to create a $k'$-tuple. 
We extend $\uparrow_k^{k'}$ to words: $\uparrow_k^{k'}(w)$ is obtained by replacing every letter $\sigma$ in $w$ with $\uparrow_k^{k'}(\sigma)$. 
Notice that if $\unzip(d\cdot e) \in \lang{\A}$, then $\unzip(\uparrow_k^{k'}(d\cdot e)) \in \lang{\A}$. 
Therefore, when the number of variables is increased, every word $w$ in the rows and columns of $T$ is replaced with $\uparrow_k^{k'}(w)$. 


\subsubsection{Learning $\nfhf$}

In the case of $\nfhf$, when the teacher returns a counterexample $S$, it holds that if $|S|> k$, then the counterexample must be positive. Indeed, assume by way of contradiction that the teacher returns a negative counterexample $S$. Then for every $k$ words $w_1,w_2,\ldots w_k$ in $S$, it holds that $zip(w_1,w_2,\ldots w_k)\in \lang{\hat\A}$, but $S\notin \hl$. Therefore, in an $\nfhf$ $\A'$ for $\hl$, there exists some word of the form $w = \zip(w_1,w_2,\ldots w_k)$ such that $w_i\in S$ for $1\leq i \leq k$, and $w\notin \lang{\hat\A'}$. As a result, $\{w_1,w_2,\ldots w_k\}\notin \hl$. Since $\zip(w_1,w_2,\ldots w_k)$ and all its sequences are in $\lang{\hat\A}$, then a smaller counterexample is $\{w_1,w_2,\ldots w_k\}$, a contradiction to the minimality of the counterexample $S$. 

In fact, if $|S|>k$ then it must be that $|S| = k+1$. Indeed, since $S$ is a positive counterexample, and $\A$ accepts all representations of subsets of size $k$ of $S$ (otherwise the teacher would return counterexample of size $k$), then there exists a subset $S'\subseteq S$ of size $k+1$ that should be represented, but is not. Therefore, $S'$ is a counterexample of size $k+1$.

As a conclusion, when a counterexample $S$ of size $k+1$ is returned, the learner updates $k\leftarrow k+1$, updates $T$ by updating every $w\in D\cup D\cdot {\hat\Sigma} \cup E$ to $\uparrow_k^{k+1}(w)$, arbitrarily selects a permutation $p$ of the words in $S$, and adds  $\zip(S)$ and all its suffixes to $E$.
In addition, it updates $D\cdot\hat\Sigma$ in accordance with the new updated $\hat\Sigma$, and fills in the missing entries accordingly.  

When $|S| \leq k$, then the counterexample is either positive or negative.
If $S$ is positive, then there exists some permutation $p$ of the words in $S$ such that $\A$ does not accept $\zip(p)$. (a permutation and not a proper sequence, otherwise there would be a smaller counterexample). The learner finds such a permutation $p$, and adds $\zip(p)$ and all its suffixes to $T$. Notice that $\zip(p)$ does not already appear in $T$, since a membership query would have returned ``yes'', and so $\A$ would have accepted $\zip(p)$.

if $S$ is negative, then $\A$ accepts all sequences of length $k$ of words in $S$, though it should not.
Then there exists a permutation $p$ of the words in $S$ that does not appear in $T$, and which $\A$ accepts. The learner then finds such a permutation $p$ and adds $\zip(p)$ and all its suffixes to $T$.

It holds, that if $p$ is a permutation of the words in $S$, and $S$ is a negative counterexample, then $\zip(p)$ should not be in $\lang{\hat\A}$ due to any other hyperword $S$, and if $S$ is a positive counterexample, then it should be in $\lang{\hat\A}$ for every $S'$ such that $S\subseteq S'$. Therefore, the above actions by the learner are valid.  

When an equivalence query succeeds, then $\A$ is indeed an $\nfhf$ for $\hl$. However, it may be the case that $\A$ is not sequence-complete, as $\hat\A$ may accept a word $w = \zip(w_1,\ldots w_k)$ but not all of its sequences. This check can be performed by the learner directly on $\hat\A$. 
Notice that $w$ does not occur in $T$, since a membership query on $w$ would return ``no''. 
Once it is verified that $\A$ is not sequence-complete, the counterexample $w$ (and all its suffixes) are added to $T$, and the procedure returns to the learning loop.
 
As we have explained above, variables are added only when necessary, and so the output $\A$ is indeed an NFH for $\hl$ with minimally many variables. 
The correctness of $\lstar$ and the minimality of the counterexamples returned by the teacher guarantee that for each $k'\leq k$, the run learns a minimal deterministic $\hat\A$ for hyperwords in $\hl$ that are represented by $k'$ variables. Therefore, a smaller $\hat\A'$ for $\hl$ does not exist, as restricting $\hat\A'$ to the first $k'$ letters in each $k$-tuple would produce a smaller underlying automaton for $k'$ variables, a contradiction. 

\begin{figure}[ht]
%\hrulefill
    \begin{center}
        \includegraphics[scale=0.5]{figures/learning_nfhf.pdf}
    \end{center}
    \caption{The first stages of learning $\lang{\A_3}$ of Figure~\ref{fig:nfh_examples}.}
    \label{fig:learning_nfhf}
%    \hrulefill
\end{figure}

\begin{example}
Figure~\ref{fig:learning_nfhf} displays the first two stages of learning $\hlang{\A_3}$ of Figure~\ref{fig:nfh_examples}.
$T_0$ displays the initial table, with $D=E = \{\epsilon\}$, and $\hat\Sigma = \{a,b,\#\}$. since $\{a\}, \{b\}$, and $\{\epsilon\}$ are all in $\hlang{\A_3}$, the initial NFH $\A$ that the learner constructs is over a single variable, and includes, according to the answers from the membership queries, a single accepting state.

Since $\hlang{\A_3}$ includes all hyperwords of size $1$, which are now accepted by $\A$, the smallest positive counterexample the teacher can return is a hyperword of size $2$, which, in the example, is $\{a,b\}$. Table $T_1$ is then obtained from $T_0$ by applying $\uparrow_1^2$, updating the alphabet $\hat\Sigma$ to $\{a,b,\#\}^2$, and updating $D\cdot\hat\Sigma$ accordingly. The table is filled by submitting membership queries. For example, for $(b,a)\in D\cdot\hat\Sigma$ and $(a,b)\in E$, the learner submits a membership query for $\{ba, ab\}$, for which the teacher returns ``no''.
\end{example}

\subsubsection{Learning $\nfhe$}

The learning process for $\nfhe$ is almost similar to the one for $\nfhf$. We briefly describe the differences. 


As in $\nfhf$, relying on the minimality of the counterexamples returned by the teacher guarantees that when a counterexample $S$ such that $|S|>k$ is returned, it is a positive counterexample. 
Indeed, assume by way of contradiction that $S$ is a negative counterexample of size $k'$. Since $\hat\A$ accepts $S$, there exists a word $\zip(w_1,\ldots w_k)$ in $\lang{\hat\A}$ such that $\{w_1,\ldots w_k\}\in S$. According to the semantics of $\exists$, if $\zip(w_1,w_2,\ldots w_k)$ in $\lang{\hat\A}$ then $S\in\hlang\A$. Since $S\notin\hl$, we have that $\{w_1,\ldots w_k\}$ is a smaller counterexample, a contradiction. 

Therefore, when the teacher returns a counterexample $S$ of size $k'>k$, the alphabet $\hat\Sigma$ is extended to $(\Sigma\cup\{\#\})^{k'}$, and the table $T$ is updated by $\uparrow_{k}^{k'}$, as is done for $\nfhf$.

If $|S|\leq k$, then $S$ may be either positive or negative. If $S$ is negative, then there exists some permutation $w$ of $S$ that is accepted by $\hat\A$. However, no such permutation is in $T$, as a membership query would have returned ``no''. Similarly, if $S$ is positive, then there exists no permutation of $S$ that $\hat\A$ accepts. In both cases, the learner chooses a permutation of $S$ and adds it, along with its suffixes, to $E$. 

As in the case of $\nfhf$, the success of an equivalence query does not necessarily imply that $\A$ is permutation-complete. 
If $\A$ is not permutation-complete, the learner finds two words $w,w'$ such that $w\in\lang{\hat\A}$ but $w'\notin\lang{\hat\A}$, and adds $w'$ as a counterexample to $E$. It holds that $w'$ was not in $T$ prior to this addition.
The procedure then returns to the learning loop.

