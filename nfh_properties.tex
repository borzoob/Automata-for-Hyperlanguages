\section{Properties of Hyperautomata and Hyper Regular Expressions}
\label{sec:nfh_properties}

%\subsection{Properties of NFH}

In this section, we consider the basic operations and decision problems for the various fragments of NFH. 
Throughout  section, we assume an NFH $\A = \tuple{\Sigma,X,Q,Q_0,\delta,F,\alpha}$ where $X = \{x_1,\ldots x_n\}$. 

%For the various complexity results, notice that  the number of variables $k$ and the number of letters $\Sigma$ dictate an alphabet size of $O(|\Sigma|^k)$ for $\hat\A$. Thus, though the state space of $\hat\A$ is of size $n$, its size may be exponential in the number of variables due to its transition relation, which may be of size $n\times |\Sigma|^k\times n$. 

\begin{theorem} \label{thm:nfh.complement}
NFH are closed under complementation.
\end{theorem}

\begin{proof}
Let $\A$ be an NFH. The NFA $\hat \A$ 
can be complemented with respect to its language over $\hat\Sigma$ to an NFA $\overline{\hat{\A}}$. 
Then, for every assignment $v:X\rightarrow S$, it holds that $\hat \A$ accepts 
$\zip(v)$ iff $\overline{\hat{\A}}$ does not accept $\zip(v)$.

Let $\overline{\alpha}$ be the quantification condition obtained from $\alpha$ by 
replacing every $\exists$ with $\forall$ and vice versa. 
Then, we can prove by induction on $\alpha$, that $\overline{\A}$, the NFH whose 
underlying automaton is $\overline{\hat{\A}}$, and whose quantification condition is 
$\overline{\alpha}$, accepts $\overline {\hlang{\A}}$.

The size of $\overline{\A}$ is exponential in $n$, due to the complementation construction for $\hat\A$.
\qed
\end{proof}


\begin{theorem} \label{thm:nfh.union}
NFH are closed under union.
\end{theorem}

\begin{proof}
The proof is similar to the standard constructive proof for NFAs. Let $\A = 
\tuple{\Sigma,X,Q,Q_0,\delta,F,\alpha}$ with underlying NFA and $\A'= 
\tuple{\Sigma',X',Q',Q'_0,\delta',F',\alpha'}$ be two NHFs. We now construct an 
NHF $\A'' = \tuple{\Sigma'',X'',Q'',Q''_0,\delta'',F'',\alpha''}$ that accepts 
$\hlang{\A} \cup \hlang{\A'}$ as follows. Let
%
\begin{itemize}
 \item  $\Sigma'' = \Sigma \cup \Sigma' \cup \{\#\}$,
 \item $X'' = X \cup X'$,
 \item $Q'' = Q \cup Q' \cup \{q_{\mathit{init}}\}$, \todo{Sarai: we allow a set of inital state, so taking the union is enough, no need for a new state}
 \item $Q''_0 = q_{\mathit{init}}$,
 \item $F'' = F \cup F'$,
 \item $\alpha'' = \alpha.\alpha'$
 
\end{itemize}
%
Transitions of $\A''$ include the following:

\begin{itemize}
 \item  We add transitions $q\xrightarrow{\epsilon} q'$, for each $q' \in 
Q''_0$, where $\epsilon$ is the null letter. 

\item For each transition $q \xrightarrow{(\sigma_{i_1}, \dots, \sigma_{i_k})} 
q' \in \delta$ (in the underlying NFA of $\A$), we include a transition $q 
\xrightarrow{(\sigma_{i_1}, \dots, \sigma_{i_k}, \#, \dots, \#)} q'$ in 
$\delta''$, where $\#$ is repeated $|\alpha'|$ number of times (i.e., the number 
of quantifiers in $\alpha$). Likewise, for each transition $q 
\xrightarrow{(\sigma_{i_1}, \dots, \sigma_{i_{k'}})} q' \in \delta'$ (in the 
underlying NFA of $\A'$), we include a transition $q \xrightarrow{(\#, \dots, 
\#, \sigma_{i_1}, \dots, \sigma_{i_{k'}})} q'$ in $\delta''$, where $\#$ is 
repeated $|\alpha|$ number of times (the number of quantifiers in $\alpha$).
\todo{Sarai: I think this is incorrect. We should not add $\#$, but every possible combination of $\Sigma\cup\{\#\}$ letters}
\end{itemize}

Now, it is straightforward to see that every hyperword in $\hlang{\A}$ or 
$\hlang{\A'}$ is accepted by $\A''$. This is because the quantifiers not 
relevant to hyperwords (e.g., $\alpha$) for a hyperword in $\hlang{\A'}$ are 
taken care of by the padded words in $(\#, \dots, \#, \sigma_{i_1}, \dots, 
\sigma_{i_{k'}})$. The same argument holds for quantifiers in $\alpha'$ and 
hyperwords in $\hlang{\A}$.\qed

\end{proof}


\begin{theorem}\label{thm:nfh.intersection}
NFH are closed under intersection.
\end{theorem}

\begin{proof}
The proof follows the closure under union and complementation. However, we can also offer a direct translation, which avoids the need to complement. 
Given two NFH $\A_1 = 
\tuple{\Sigma,X,Q,Q_0,\delta_1,F_1,\alpha_1}$ and $\A_2= 
\tuple{\Sigma,Y,P,P_0,\delta_2,F_2,\alpha_2}$, we construct an NFH $\tuple{\Sigma,X\cup Y, (Q\cup \{q\} \times P\cup \{p\}, P_0\times Q_0, \delta, (F_1\cup\{q\}) \times (F_2\cup\{p\}), \alpha}$, where 
$\alpha = \alpha_1\alpha_2$, and $\delta$ is defined as follows. 
\begin{itemize}
    \item For every $(q_1\xrightarrow {(\sigma_1,\ldots \sigma_k)} q_2)\in \delta_1$ and every $(p_1\xrightarrow {(\sigma'_1\ldots \sigma'_{k'}) }p_2)\in\delta_2$, we have 
$$((q_1,p_1)\xrightarrow{(\sigma_1\ldots \sigma_k,\sigma'_1,\ldots \sigma'_{k'})} (q_2,p_2))\in\delta$$
\item For every $q_1\in F_1, (p_1 \xrightarrow{(\sigma'_1,\ldots \sigma'_{k'})} p_2)\in \delta_2$ we have $$((q_1,p_1)\xrightarrow {(\overbrace{\#,\ldots\#}^{k},\sigma'_1,\ldots \sigma'_{k'})}(q,p_2)), ((q,p_1)\xrightarrow {(\overbrace{\#,\ldots\#}^{k},\sigma'_1,\ldots \sigma'_{k'})}(q,p_2)) \in \delta$$
\item For every $(q_1\xrightarrow {(\sigma_1,\ldots \sigma_k)} q_2)\in \delta_1$ and $p_1\in F_2$, we have $$((q_1,p_1)\xrightarrow {(\sigma_1,\ldots \sigma_{k},\overbrace{\#,\ldots\#}^{k'})}(q_2,p)),((q_1,p)\xrightarrow {(\sigma_1,\ldots \sigma'_{k},\overbrace{\#,\ldots\#}^{k'})}(q_2,p))\in \delta$$
\end{itemize}

Intuitively, the role of $q,p$ is to keep reading $(\#)^k$ and $(\#)^{k'}$ after the word read by $\hat\A_1$ or $\hat\A_2$, respectively, has ended. 

The NFH $\hat\A$ simultaneously reads two words $\zip(w_1,\ldots w_k)$ and $\zip(w'_1,w'_2,\ldots w'_{k'})$ that are read along $\hat\A_1$ and $\hat\A_2$, respectively, and accepts only if both words are accepted. The correctness follows from the fact that for $v:(X\cup Y)\rightarrow S$, we have that $\zip(v)$ is accepted by $\hat\A$ iff $\zip (v|_X)$ and $\zip(v|_Y)$ are accepted by $\hat\A_1$ and $\hat\A_2$, respectively. 

This construction is polynomial in the sizes of $\A_1$ and $\A_2$.
\end{proof}


We now turn to study various decision problems for NFH. We begin with the nonemptiness problem: given an NFH $\A$, is $\hlang{\A} = \emptyset$? We show that while the problem is in general undecidable for NFH, it is decidable for most fragments that we consider. 


\begin{theorem}
The nonemptiness problem for NHF is undecidable.
\end{theorem}

\begin{proof}

We mimic the proof in \cite{fh16}, with a reduction from the {\em post correspondence problem (PCP)}.
%
The PCP is as follows.
%
Fix an alphabet $\alphabet_{\mathit{pcp}}$ of size at least two. \todo{Sarai: we can just use $a,b$}
%
Given is a collection of dominos of the form:
$$\Big[\frac{w}{v} \Big] = \bigg\{\Big[\frac{w_1}{v_1} \Big], 
\Big[\frac{w_2}{v_2} \Big],\dots, \Big[\frac{w_k}{v_k} \Big] \bigg\}$$
where for all $i \in [1, k]$, we have $v_i,w_i \in \alphabet_{\mathit{pcp}}^*$.
%
The decision problem consists of whether there exists a finite sequence of
dominos of the form
$$\Big[\frac{w_{i_1}}{v_{i_1}} \Big]\Big[\frac{w_{i_2}}{v_{i_2}} \Big] \cdots 
\Big[\frac{w_{i_m}}{v_{i_m}} \Big]$$
where each index $i_j \in [1, k]$, such that the upper and lower finite strings 
of the dominos are equal, i.e.,
$$v_{i_1}v_{i_2}\cdots{}v_{i_m} = w_{i_1}w_{i_2}\cdots{}w_{i_m}.$$
%
For some collection of dominos, it is impossible to find a match. An example is 
the following:
$$\Big[\frac{w}{v} \Big] = \bigg\{\Big[\frac{abc}{ab} \Big], \Big[\frac{ca}{a} 
\Big], \Big[\frac{acc}{ba} \Big]\bigg\}$$

\stam{
First, we map an arbitrary instance of the PCP 
$[\frac{w}{v}]=\{[\frac{w_2}{v_2}], [\frac{w_2}{v_2}], \dots, 
[\frac{w_k}{v_k}]\}$ to an NHF.
}

Given an instance $C$ of PCP, we encode a solution as a word $w_{sol}$ over $\alphabet = \{\frac{\sigma}{\sigma'}| \sigma,\sigma'\in\{a,b,{\dot a},{\dot b}, \$\}\}$.

For a word $w = \frac{\sigma_1}{\sigma'_1}   \frac{\sigma_2}{\sigma'_2}  \cdots  \frac{\sigma_n}{\sigma'_n}  $, we call $\sigma_1\cdots \sigma_n$ the {\em upper part} of $w$, and the word $\sigma'_1\cdots \sigma'_n$ the {\em lower part} of $w$. 

Intuitively, $\dot{\sigma}$ marks the beginning of a new tile, and $\$$ marks the end of a sequence of the upper or lower parts of the dominos.

$w_{sol}$ encodes a legal solution if the following conditions are met.
\begin{enumerate}
    \item For every $ \frac{\sigma}{\sigma'}  $ that occurs in $w_{sol}$, it holds that $\sigma,\sigma'$ represent the same domino letter (both $a$ or both $b$).
    \item and the number of dotted letters in the upper part of $w_{sol}$ is equal to the number of dotted letters in the lower part of $w_{sol}$.
    \item $w_{sol}$ starts with two dotted letters, and the word $u_i$ between the $i$'th and $i+1$'th dotted letters in the upper part of $w_{sol}$, and the word $v_i$ between the corresponding dotted letters in the lower part of $w_{sol}$ are such that $ \frac{u_i}{v_i}  \in C$.
\end{enumerate}

%A {\em partial solution} is a word $w\in \Sigma^*$ that only satisfies conditions $2$ and $3$, and in which $\$$ does not occur in either part (lower or upper) before an $a$ or $b$ representation in this part.
We call a word that represents the removal of the first $k$ tiles from $w_{sol}$ a {\em partial solution}, denoted by $w_{sol,k}$.
Note that the upper and lower parts of $w_{sol,k}$ are not necessarily of equal lengths (in terms of $a$ and $b$ sequences), since the upper and lower parts of a domino tile are not necessarily of equal lengths, and so $\$$ is used to pad the end of the encoding in the shorter of the parts. 

We construct an NFH $\A$, which, intuitively, expresses the following idea: $(1)$ there exists an encoded solution $w_{sol}$, and $(2)$ For every $w_{sol,k}$ in a hyperset $S$ accepted by $\A$, the word $w_{sol,k+1}$ is also in $S$. 

$\hlang{\A}$ is then the set of all hyperwords that consist of a solution $w_{sol}$, and all its suffixes obtained by removing a prefix of tiles from $w_{sol}$.  

For example, if the set of tiles is 
$$ \{ [\frac{ab}{b}], [\frac{ba}{a}],[\frac{a}{aba}] \} $$
Then a possible solution is 
$$ [\frac{a}{aba}][\frac{ba}{a}] [\frac{ab}{b} ]  $$
and so a matching hyperword $S$ that is accepted by $\A$ is
$$ S = \{  w_{sol} = \frac{\dot a}{\dot a}   \frac{\dot b}{b}   \frac{a}{a}   \frac{\dot a}{\dot a}   \frac{b}{\dot b} ,
w_{sol,1} = \frac{\dot b}{\dot a}\frac{a}{\dot b}\frac{\dot a}{\$}\frac{b}{\$}, w_{sol,2} = \frac{\dot a}{\dot b}\frac{b}{\$}, w_{sol,3} = \epsilon
\} $$

Thus, the acceptance condition of $\A$ is $\alpha = \forall x_1\exists x_2 \exists x_3$, where $x_1$ is to be assigned a partial solution $w_{sol,k}$, the variable $x_2$ is to be assigned $w_{sol,k+1}$, and $x_3$ is to be assigned $w_{sol}$.  

During a run on a hyperword $S$ and an assignment $v:\{x_1,x_2,x_3\}\rightarrow S$, the NFH $\A$ checks that the upper and lower letters of $w_{sol}$ all match.
In addition, it checks that $v(x_2)$ is obtained from $v(x_1)$ by removing the first tile. 

It does so by remembering, using the states, the  




\todo{TBC}
\end{proof}



\begin{theorem} \label{thm:nfhe.nfhf.nonemptiness}
The nonemptiness problem for $\nfhe$ and $\nfhf$ is NL-complete.
\end{theorem}

\begin{proof}
We begin with $\nfhe$. Let $\A_\exists$ be an $\nfhe$ over a set of variables $X 
= \{x_1,x_2,\ldots x_k\}$. 
Let $w$ be a word over $(\Sigma\cup \#)^k$. We say that $w$ is {\em legal} if 
$w=\zip(u_1,\ldots u_k)$ for $u_1,u_2,\ldots u_k \in \Sigma^*$. 
Note that $w$ is legal if there is no $w[i]$ in which there is a $\#$ that is 
followed by some $\sigma\in \Sigma$. 
We claim that $\A_\exists$ is nonempty iff $\hat\A_\exists$ accepts some legal 
word $w$. The first direction is trivial. For the second direction, let 
$w\in\lang{\hat\A_\exists}$, and let $S$ be the set of words in $\unzip(w)$. By 
assigning $v(x_i) = w[i]$ for every $x_i\in X$, we get $\zip(v) = w$, and 
according to the semantics of $\exists$, we have that $\A_\exists$ accepts $S$. 
To check whether $\hat\A_\exists$ accepts a legal word, we can run a 
reachability check on-the-fly, while advancing from a letter $\sigma$ to the 
next letter $\sigma'$ only if $\sigma'$ contains $\#$ in all the positions in 
which $\sigma$ contains $\#$. 
While each transition $T = q\xrightarrow{(\sigma_1,\ldots \sigma_n)} p$ in $\hat\A$ is of size $k$, we can encode $t$ as a set of size $k$ of encodings of transitions of type $q \xrightarrow {\sigma_i} p$, with a binary encoding of $p,q,\sigma_i$, as well as $i,t$, where $t$ marks the index of $T$ within the set of transitions of $\hat\A$. 
Now, let $\A_\forall$ be an $\nfhf$ over $X$. We claim that $\A_\forall$ is 
nonempty iff $\A_\forall$ accepts a hyperword of size $1$. 
For the first direction, let $S\in\lang{\A_\forall}$. Then, by the semantics of 
$\forall$, we have that for every assignment $v:X\rightarrow S$, it holds that 
$\zip(v)\in\hat{\A_\forall}$. Let $u\in S$, and let $v_u(x_i) = u$ for every 
$x_i\in X$. Then, in particular, $\zip(v_u)\in\lang{\hat{\A_\forall}}$. We have 
then that for every assignment $v:X\rightarrow \{u\}$ (which consists of the 
single assignment $u_v$), it holds that $\hat{\A_\forall}$ accepts $\zip(v)$, 
and therefore $\A_\forall$ accepts $\{u\}$. 
The second direction is trivial. 

To check whether $\A_\forall$ accepts a hyperword of size $1$, we restrict the 
alphabet of $\hat\A_\forall$ to $k$-tuples of the form $(\sigma,\sigma,\ldots 
\sigma)$ for $\sigma\in \Sigma$, by removing all other transitions. Then, we 
check whether the remaining NFA is nonempty. 

Since the nonemptiness problem for NFA is NL-complete, the results for both 
fragments follow. 
\end{proof}

We now show that the nonemptiness problem for $\nfhef$ is decidable. We first 
present some terms and notations.  

Consider a tuple $t = (t_1,t_2,\ldots t_k)$ of items. 
A {\em sequence} of $t$ is a tuple $(t'_1, t'_2,\ldots t'_k)$, where 
$t'_i\in\{t_1,\ldots t_k\}$ for every $1\leq i \leq k$. 
A {\em permutation} of $t$ is a reordering of the elements of $t$. 
We extend the notion of permutation to words over $k$-tuples, and to 
assignments, as follows. 
Let $w = \zip(w_1,\ldots w_k)$ be a word over $k$-tuples, and let $\zeta = 
(i_1,i_2,\ldots i_k)$ be a sequence (permutation) of $(1,2,\ldots, k)$. The word $w_\zeta$, 
defined as $\zip(w_{i_1}, w_{i_2}, \ldots w_{i_k})$ is a sequence (permutation) of $w$.
Finally, let $v$ be an assignment from a set of variables $\{x_1,x_2,\ldots 
x_k\}$ to a hyperword $S$. The assignment $v_\zeta$, defined as $v_\zeta(x_j) = 
v(x_{i_j})$ for every $1\leq i,j \leq k$, is a sequence (permutation) of $v$. 
\todo{Sarai: move this elsewhere? since we use these terms also in learning}

We now show that if an $\nfhef$ is nonempty, then we can bound the size of a 
hyperword that it accepts.

\begin{lemma}\label{lemma:nfhef.nonempty}
Let $\A$ be an $\nfhef$ with a quantification condition $\alpha = \exists x_1,\ldots \exists x_m \forall 
x_{m+1}\ldots \forall x_k$, where $1 \leq m < k$. 
Then $\A$ is nonempty iff $\A$ accepts a hyperword of size at most $m$.
\end{lemma}

\begin{proof}
Let $S\in\hlang{\A}$. Then according to the semantics of the quantifiers, there 
exist $w_1,\ldots w_m \in S$, such that for every 
assignment $v:X\rightarrow S$ in which $v(x_i) = w_i$ for every $1\leq i\leq m$, 
it holds that $\hat\A$ accepts $\zip(v)$.
Let $v:X\rightarrow S$ be such an assignment.
Then $\hat\A$ accepts 
$\zip(v_\zeta)$ for every sequence $\zeta$ of the form $(1,2,\ldots m,i_1,i_2,\ldots 
i_{m-k})$. 
In particular, it holds for such sequences in which 
$1\leq i_j\leq m$ for every $1\leq j \leq k-m$,
that is, sequences in which the last $k-m$ variables are assigned words from 
the set of words that are assigned to the first $m$ variables. Therefore, again 
by the semantics of the quantifiers, we have that $\{v(x_1),\ldots v(x_m)\}$ is 
in $\hlang{\A}$.
The second direction is trivial.
\end{proof}

We now use Lemma~\ref{lemma:nfhef.nonempty} to describe a decision procedure for 
the nonemptiness of $\nfhef$. 

\begin{theorem}\label{thm:nfhef.nonemptiness}
The nonemptiness problem for an $\nfhef$ $\A$ can be decided in space that is polynomial in the size of $\hat\A$.
\end{theorem}

\begin{proof}
Let $\A$ be as described in Lemma~\ref{lemma:nfhef.nonempty}.
According to Lemma~\ref{lemma:nfhef.nonempty}, 
$\A$ is nonempty iff there exists a word $w\in\lang{\hat\A}$ such that
$\hat\A$ accepts $w_\zeta$ for every sequence $\zeta$ of the form $(1,2,\ldots m, 
i_1,i_2,\ldots i_{k-m})$, where $1\leq i_j\leq m$ for every $1\leq j \leq k-m$.
Indeed, by the semantics of the quantifiers, it holds that $\A$ accepts 
$\{w[1],w[2],\ldots w[m]\}$.
We call $w$ a {\em witness to the nonemptiness of $\A$}.

We construct an NFA $A$ based on $\hat\A$ that is nonempty iff $\hat\A$ accepts 
a witness to the nonemptiness of $\A$.
Let $\Gamma$ be the set of all sequences of the above form.
For every sequence $\zeta = (i_1,i_2,\ldots i_k)$ in $\Gamma$, we construct 
an NFA $A_\zeta = \tuple{ \hat\Sigma,Q,Q_0,\delta_\zeta,F}$, where for every 
$q\xrightarrow{(\sigma_{i_1},\sigma_{i_2},\ldots \sigma_{i_k})} q'$ in $\delta$, 
we have $q\xrightarrow{(\sigma_1,\sigma_2,\ldots \sigma_k)} q'$ in 
$\delta_\zeta$.
Intuitively, $A_{\zeta}$ runs on every word $w$ the same way that $\hat\A$ runs 
on $w_\zeta$. Therefore, $\hat\A$ accepts a witness $w$ to the nonemptiness of 
$\A$ iff $w\in\lang{A_\zeta}$ for every $\zeta\in\Gamma$. 

We define $A = \bigcap_{\zeta\in\Gamma} A_\zeta$.
Then $\hat\A$ accepts a witness to the nonemptiness of $\A$ iff $A$ is 
nonempty. 
Since $|\Gamma| = m^{k-m}$, the state space of $A$ is of size $O(n^{m^{k-m}})$, and its alphabet is of size $|\hat\Sigma|$. 

Notice that for $\A$ to be nonempty, $\hat\Sigma$ must be of size at least $|(\Sigma\cup {\#})|^{(k-m)}$, to account for all the permutations of letters in the words assigned to the variables under $\forall$ quantifiers (otherwise, we can immediately return ``empty''). Therefore, $|\hat\A|$ is $O(n\times \Sigma^k)$. 
We then have that the size of $A$ is $O(|\hat \A|^k)$.  

If the number $k-m$ of $\forall$ quantifiers is fixed, then $m^{k-m}$ is polynomial in $k$. However, now $|\hat\A|$ may be polynomial in $n,k,|\Sigma|$, and so in this case as well, the size of $A$ is $O(|\hat A|^k)$. 

Since the nonemptiness problem for NFA is NL-complete, the problem for $\nfhef$ can be solved in space that is polynomial in $|{\hat\A}|$. 

\end{proof}

\begin{theorem}
The nonemptiness problem for $\nfhef$ is  PSPACE-complete.
\end{theorem}
\begin{proof}
The upper bound follows from Theorem~\ref{thm:nfhef.nonemptiness}. 
For the lower bound, we show a reduction from a polynomial version of the {\em tiling problem}, %~\cite{Boas97}, 
defined as follows.
We are given a finite set $T$ of tiles, two relations $V \subseteq T \times T$ and $H \subseteq T \times T$,
an initial tile $t_0$, a final tile $t_f$, and a bound $n>0$.
We have to decide whether there is some $m>0$ and a tiling of a $n \times m$-grid such that
(1) The tile $t_0$ is in the bottom left corner and the tile $t_f$ is in the top right corner,
(2) A horizontal condition: every pair of horizontal neighbors is in $H$, and
(3) A vertical condition: every pair of vertical neighbors is in $V$.

When $n$ is given in unary, the problem is known to be PSPACE-complete.

Given an instance $C$ of the tiling problem, We construct an $\nfhef$ $\A$ that is nonempty iff $C$ has a solution. 
We encode a solution to $C$ as a word $w_{sol} =w_1\cdot w_2\cdot w_m\$$ over $T$, where $w_i$, of the form $1\cdot t_{1,i}\cdot 2 \cdot t_{2,i},\ldots n\cdot t_{n,i}$, describes the contents of row $i$. 

To check that $w_{sol}$ indeed encodes a solution, we need to make sure that:
\begin{enumerate}
\item $w_1$ begins with $t_0$ and $w_m$ ends with $t_f\$$,
\item Within every $w_i$, it holds that $(t_{j,i},t_{j+1,i})\in H$,
    \item $w_i$ is of the correct form,
    \item For $w_i,w_{i+1}$, it holds that $(t_{j,i}, t_{j,i+1})\in V$ for every $1\leq j\leq n$
\end{enumerate} 

Verifying items $1-3$ is easy via an NFA of size $O(n|H|)$.  
The main obstacle is item $4$. 

We describe an $\nfhef$ $\A = \tuple{T\cup \{0,1,2,\ldots n,\$\}, \{y_1,y_2,y_3,x_1,\ldots x_{\log(n)}\}, Q, \{q_0\},\delta, F, \alpha}$ that is nonempty iff there exists a word that satisfies items $1-4$.
The quantification condition $\alpha$ is $\exists y_1\ldots \exists y_3\forall x_1 \ldots \forall x_{\log(n)}$.
$\A$ only proceeds on letters whose first three positions are of the type $(r,0,1)$, where $r\in T\cup\{1,\ldots n,\$\}$. Notice that this means that it requires the existence of the words $0^{|w_{sol}|}$ and $1^{|w_{sol}|}$ (the $0$ word and $1$ word, henceforth).
$\A$ makes sure that the word assigned to $y_1$ matches a correct solution w.r.t. items $1-3$ described above.  
We proceed to describe how to handle the requirement for $V$. 
We need to make sure that for every position $j$ in a row, the tile in position $j$ in the next row matches the current one w.r.t. $V$. We can use a state $q_j$ to remember the tile in position $j$, and compare it to the tile in the next occurrence of $j$. The problem is avoiding having to check all positions simultaneously, which would require exponentially many states. To this end, we use $\log(n)$ copies of the $0$ and $1$ words to form a binary encoding of the position $j$ that is to be remembered. The $\log(n)$ $\forall$ conditions make sure that every position within $1-n$ is checked.  

We limit the checks to words in which $x_1,\ldots x_{\log(n)}$ are the $0$ or $1$ words, by having $\hat\A$ accept every word in which there is a letter that is not over $0,1$ in positions $4,\ldots \log(n)+3$. Notice that this takes care of accepting all cases in which the word assigned to $y_1$ is also assigned to one of the $x$ variables. 

To check that $x_1,\ldots x_{\log(n)}$ are the $0$ or $1$ words, $\hat\A$ checks that the values in positions $4$ to $\log(n)+3$ remain constant throughout the run. 
In these cases, upon reading the first letter, $\hat\A$ remembers the value $j$ that is encoded by the constant assignments to $x_1,\ldots x_{\log(n)}$ in a state, and makes sure that throughout the run, the tile that occurs in the assignment $y_1$ in position $j$ in the current row matches the tile in position $j$ in the next row. 

We can construct a similar reduction for the case that the number of $\forall$ quantifiers is fixed: instead of encoding the position by $\log(n)$ bits, we can directly specify the position by a word of the form $j^*$, for every $1\leq j\leq n$. 
Accordingly, we construct an $\nfhef$ over $\{x, y_1,\ldots y_{n},z\}$, with a quantification condition $\alpha = \exists x\exists y_1 \ldots \exists y_{n}\forall z$. The NFA $\hat\A$ advances only on letters whose assignments to $y_1,\ldots y_n$ are always $1,2,\ldots n$, respectively, and checks only words assigned to $z$ that are some constant $1\leq j\leq n$. Notice that the fixed assignments to the $y$ variables leads to $\hat\Sigma$ of polynomial size.  
In a hyperword accepted by $\A$, the word assigned to $x$ is $w_{sol}$, and the word assigned to $z$ specifies which index should be checked for conforming to $V$.
\end{proof}

We turn to study the membership problem for NFH: given an NFH $\A$ and a hyperword $S$, is $S\in\hlang{\A}$? 
When $S$ is finite, the set of assignments from $X$ to $S$ is finite, and so the problem is decidable. We call this case the {\em finite membership problem}. 

\begin{theorem}\label{thm:nfh.membership.finite}
Let $\A$ be an NFH and $S$ be a finite hyperword over $\Sigma$. Then it can be decided whether $S\in \hlang{\A}$ in space that is polynomial in $k$, and logarithmic in $|\hat\A|, |S|$.
\end{theorem}

\begin{proof}
As discussed in the proof of Theorem~\ref{thm:nfhef.nonemptiness}, the size of $\hat\Sigma$ must be exponential in the number of $\forall$ quantifiers in $\alpha$, and therefore is exponential in $k$. 
We can decide the membership of $S$ in $\hlang{\A}$ by iterating over all assignments of the type $X\rightarrow S$. For every such assignment $v$, we construct $\zip(v)$ and run $\hat\A$ on $\zip(v)$, on-the-fly. 
\end{proof}

The exponential size of $\hat\A$ is derived from the number of $\forall$ quantifiers. When the number of $\forall$ quantifiers is fixed, then $\hat\Sigma$ is not necessarily exponential in $k$. 

\begin{theorem}
The finite membership problem for NFH with a fixed number of $\forall$ quantifiers is NP-complete.
\end{theorem}

\begin{proof}
As in the proof of Theorem~\ref{thm:nfh.membership.finite}, we can iterate over all assignments to the variables under $\forall$, while guessing assignments to the variables under $\exists$. Since the number of $\forall$ variables is fixed, the number of iterations polynomial in $k,|S|$. For each such assignment $v$, checking whether $\zip(v)\in\lang{\hat\A}$ can be done on-the-fly. 

Notice that now we cannot claim that the size of $\hat\Sigma$ is exponential in $k$, since $m$ is fixed, and a much smaller set of letters may suffice for the $\exists$ conditions. Therefore, the described solution provides membership in $NP$. 

We show NP-hardness by a reduction from the Hamiltonian cycle problem. Given a graph $G = \tuple{V,E}$ where $V = \{v_1,v_2,\ldots v_n\}$ and $|E|=m$, we construct an $\nfhe$ $\A$ over $\{0,1\}$ with $n$ states, $n$ variables, $\hat\Sigma$ of size $m$, and a hyperword $S$ of size $n$, as follows. $S = \{w_1,\ldots w_n\}$, where $w_i$ is the word in $\{0,1\}$ in which all letters are $0$ except for the $i$'th. 
The structure of $\hat\A$ is identical to that of $G$, and we set $Q_0 = F = \{v_1\}$. For the transition relation, for every $(v_i,v_j)\in E$, we have $(v_i, \sigma_i,v_j)\in \delta$, where $\sigma_i$ is the letter over $\{0,1\}^n$ in which all positions are $0$ except for position $i$. 
Intuitively, the $i$'th letter in an accepting run of $\hat\A$ marks traversing $v_i$. The assignment of $x_i$ to $w_j$ means that the $j$'th step of the run traverses $v_i$. Since the words in $w$ make sure that every $v\in V$ is traversed exactly once, and that the run on them is of length $n$, we have that $\A$ accepts $S$ iff there exists some permutation $p$ of the words in $S$ such that $p$ matches a Hamiltonian cycle in $G$.
\end{proof}

When $S$ is infinite, it may still be finitely represented. 
We now address the problem of deciding whether a regular language $\cal L$ 
(given as an NFA) is accepted by an NFH. We call this {\em the regular 
membership problem for NFH}. We show that this problem is decidable for the 
entire class of NFH. 

\begin{theorem}
The regular membership problem for NFH is decidable.
\end{theorem}

\begin{proof}
Let $A = \tuple{\Sigma, P, P_0, \rho,F}$ be an NFA, and let $\A = \tuple{\Sigma, 
\{x_1,\ldots x_k\}, Q, Q_0, \delta, {\cal F},\alpha}$ be an NFH. 

First, we construct an NFA $A'=\tuple{\Sigma\cup\{\#\}, P', P'_0, \rho', F'}$
by extending the alphabet of $A$ to $\Sigma\cup\{\#\}$, adding a new and 
accepting state $p_f$ to $P$ with a self-loop labeled by $\#$, and transitions 
labeled by $\#$ from every $q\in F$ to $p_f$. 
The language of $A'$ is then $\lang{A}\cdot \#^*$.  
We describe a recursive procedure (iterating over $\alpha$) for deciding 
whether $\lang{A}\in\hlang{\A}$.

For the case that $k=1$, it is easy to see that if $\alpha = \exists x_1$, then 
$\lang{A}\in\hlang{\A}$ iff $\lang{A}\cap \hlang{\hat{\A}} \neq \emptyset$.
Otherwise, if $\alpha = \forall x_1$, then $\lang{A}\in\hlang{\A}$ iff 
$\lang{A}\notin \hlang{\overline{\A}}$, where $\overline{\A}$ is the NFH for 
$\overline{\hlang{\A}}$ described in Theorem~\ref{thm:nfh.complement}. 
Notice that the quantification condition for $\overline{\A}$
is $\exists x_1$, and so this conforms to the base case.

For $k>1$, we construct a sequence of NFH $\A_1, \A_2, \ldots, \A_k$. We set 
$\A_1 = \A$, and
construct $\A_{i+1}$ given $\A_i = \tuple{\Sigma, \{x_i,\ldots x_k\}, Q_i, 
Q^0_i, \delta_i, {\cal F}_i,\alpha_i}$, as follows. 

The set of variables of $\A_{i+1}$ is $\{x_{i+1},\ldots x_k\}$, and the 
quantification condition $\alpha_{i+1}$ is $\quant_{i+1}x_{i+1}\cdots 
\quant_kx_k$, for $\alpha_i = \quant_ix_i \quant_{i+1}\cdots \quant_kx_k$. 
The set of states of $\A_{i+1}$ is $ Q_i\times P'$, and the set of initial 
states is $Q_i^0\times P_0$. The set of accepting states is ${\cal F}_i\times 
F'$.
For every 
$(q\xrightarrow{\sigma_i,\ldots,\sigma_k}q')\in\delta_i$ and every 
$(p\xrightarrow{\sigma_i}p')\in \rho$, we have 
$((q,p)\xrightarrow{\sigma_{i+1},\ldots \sigma_k}(q',p'))\in\delta_{i+1}$. 
Then, $\hat\A_{i+1}$ accepts a word $\zip(u_1,u_2,\ldots u_{k-i})$ iff there 
exists a word $u\in \lang{A}$, such that $\hat\A_{i}$ accepts 
$\zip(u,u_1,u_2,\ldots u_{k-i})$. 

We first consider the case that $\quant_i = \exists$. 
Let $v:\{x_{i},\ldots x_k\}\rightarrow \lang{A}$.
Then $\lang{A}\models _v \alpha_i\A_i$ iff there exists $w\in \lang{A}$ such 
that $\lang{A}\models_{v[x_i\rightarrow w]} \alpha_{i+1},A_i$.
For an assignment $v':\{x_{i+1},\ldots x_k\}\rightarrow \lang{A}$, it holds that 
$\zip(v')$ is accepted by $\hat{A}_{i+1}$ iff there exists a word $w\in 
\lang{A}$ such that $\zip(v)\in\lang{\hat{A}_i}$, where $v$ is obtained from $v'$ 
by setting $v(x_i) = w$. 
Therefore, we have that $\lang{A}\models_{v[x_i\rightarrow w]}\alpha_{i+1},\A_i$ 
iff $\lang{A}\models_{v'} \alpha_{i+1}, \A_{i+1}$, that is, 
$\lang{A}\in\hlang{\A_i}$ iff $\lang{A}\in\hlang{\A_{i+1}}$. 

If $\quant_i = \forall$, then we have that $\lang{A}\in \lang{A_i}$ iff 
$\lang{A}\notin \overline{\hlang{\A_i}}$. We construct $\A'_{i}$ for 
$\overline{\hlang{\A_i}}$ as described in Theorem~\ref{thm:nfh.complement}. The 
quantification condition of $\A'_{i}$ then begins with $\exists_i x_i$, and we 
can apply the previous case, and construct $\A_{i+1}$ w.r.t. $\A'_{i}$, to check for non-membership.

Every $\forall$ quantifier requires complementation, which is exponential in $n$. Therefore, in the worst case, the complexity of this algorithm is $O(2^{2^{...^{n|P|}}})$, where the tower is of height $k$. If the number of $\forall$ quantifiers is fixed, then the complexity is $O(n|P|^k)$. 

\end{proof}


\begin{theorem}
The inclusion problems of $\nfhe$ and $\nfhf$ in $\nfhe$ and $\nfhf$ and of $\nfhef$ in $\nfhe$ and $\nfhf$ are PSPACE-complete.
\end{theorem}
\begin{proof}
The lower bound follows from the lower bound for the inclusion problem for NFAs. 

For the upper bound, first notice that complementing an $\nfhf$ yields an $\nfhe$, and vice versa. 
Consider two NFH $\A_1$ and $\A_2$, where $\A_1,\A_2\in\nfhf,\nfhe$. 
Then $\hlang{\A_1}\subseteq\hlang{\A_2}$ iff $\hlang{\A_1}\cap\overline{\hlang{\A_2}}  =  \emptyset$. 
We can use the constructions in Theorems~\ref{thm:nfh.complement} and~\ref{thm:nfh.intersection} to compute a matching NFH $\A = \A_1\cap\overline{\A_2}$, and check its nonemptiness.
The complementation construction is exponential in $n_2$, the number of states of $\A_2$, and the intersection construction is polynomial in $|\A_1|, |\overline{\A_2}|$. 

If $\A_1\in\nfhe$ and $\A_2\in\nfhf$ or vice versa, then $\A$ is an $\nfhe$ or $\nfhf$, respectively, whose nonemptiness can be decided in space that is logarithmic in $|\A|$. 

Now, consider the case where $\A_1$ and $\A_2$ are both $\nfhe$ or both $\nfhf$.
It follows from the proof of Theorem~\ref{thm:nfh.intersection}, that for two NFH $\A,\A'$, the quantification condition of $\A\cap\A'$ may be any interleaving of the quantification conditions of $\A$ and $\A'$. Therefore, if $\A_1,\A_2\in\nfhe$ or $\A_1,\A_2\in\nfhf$, we can construct $\A$ to be an $\nfhef$. 
This is also the case when $\A_1\in\nfhef$ and $\A_2$ is an $\nfhe$ or $\nfhf$.
According to Theorem~\ref{thm:nfhef.nonemptiness}, the nonemptiness of $\nfhef$ is 
Since nonemptiness of $\nfhef$ is decidable (Theorem~\ref{thm:nfhef.nonemptiness}), we are done. 
\end{proof}

\iffalse

\subsection{Properties of NBH}

Since NBWs are closed under the Boolean operations, and their nonemptiness and inclusion problems are decidable (and are NL-complete and PSPACE-complete, respectively), the results of Section~\ref{sec:nfh_properties} can all be lifted to the case of hyper-infinite words. 
To summarize, we have the following. 

\begin{theorem}
\begin{itemize}
    \item NBH are closed under the Boolean operations.
    \item The nonemptiness problem for NBH$_\exists$ is NL-complete.
    \item The nonemptiness problem for NBH$_{\exists\forall}$ is decidable in EXPSPACE.
    \item The nonemptiness problem for NFH$_{\forall\exists}$ is undecidable.
    \item The $\omega$-regular membership problem for NBH is decidable.\todo{this is just a bit trickier, since we need to be more careful with the accepting states}
\end{itemize}
\end{theorem}

\fi