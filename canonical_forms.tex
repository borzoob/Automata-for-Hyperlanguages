
\subsection{Canonical Forms for $\nfhf$ and $\nfhe$}

Throughout this section, we discuss NFH over an alphabet $\Sigma$ with a set of variables $X = \{x_1,x_2,\ldots x_k\}$.

\subsubsection{A canonical form for $\nfhf$}

We begin with a canonical form for $\nfhf$.

Let $\A$ be an $\nfhf$. 
We say that an $\nfhf$ $\A'$ is {\em sequence complete} if for every word $w$, it holds that $\hat \A'$ accepts $w$ iff it accepts every sequence of $w$. 

\begin{lemma}
Let $\A'$ be a sequence complete $\nfhf$ over $\Sigma$ and $X$, such that 
$\hat\A'$ accepts $w$ iff $\hat\A$ accepts every sequence of $w$. 
Then $\lang{\A} =\lang{\A'}$.
\end{lemma}
\begin{proof}
For the first direction, since $\lang{\hat\A'}\subseteq \lang{\hat\A}$, we have $\lang{\A'}\subseteq \lang{\A}$.
For the second direction, let 
$S\in\lang{A}$. Then for every $v:S\rightarrow X$, it holds that $\zip(v)\in\lang{\hat\A}$. Also,   $\zip(v')\in\lang{\hat\A}$ for every sequence $v'$ of $v$. Then $\zip(v)$ and all its sequences are in $\lang{\hat\A'}$. Since this holds for every $v:X\rightarrow S$, we have that $S\in\lang{A'}$.
\end{proof}


\begin{lemma}
Every $\nfhf$ $\A$ has an equivalent sequence-complete $\nfhf$ $\A'$ over the same set of variables. 
\end{lemma}
\begin{proof}
To construct $\A'$ given $\A$, we use a similar construction to the one presented in the proof of Theorem~\ref{thm:nfhef.nonemptiness}. Essentially, for every sequence $\zeta$ of $(1,2,\ldots k)$, we construct an NFA $A_\zeta$, in which every run on a word $w$ matches a run of $\hat\A$ on $w_\zeta$. 
The $\nfhf$ $\A'$ is then obtained from $\A$ by replacing the underlying NFA with $\bigcap_{\zeta\in\Gamma} A_\zeta$, where $\Gamma$ is the set of sequences of $(1,2,\ldots k)$.
\end{proof}

\begin{lemma}
Let $\A_1$ and $\A_2$ be two sequence-complete $\nfhf$ over the same set of variables $X$.
Then $\lang{\A_1}=\lang{\A_2}$ iff  $\lang{\hat\A_1} = \lang{\hat \A_2}$.
\end{lemma}
\begin{proof}
For the first direction, let $w\in\lang{\hat\A_1}$.
Since $\A_1$ is sequence-complete, then $w'\in\lang{\hat\A_1}$ for every sequence $w'$ of $w$. Then, by the semantics of the $\forall$ quantifier, we have that $\unzip(w)\in\lang{\A_1}$ Therefore, $\unzip(w)\in\lang{\A_2}$, and so $w$ (and all its sequences) are in $\lang{\hat\A_2}$. A similar argument can be made to show that for every $w\in\lang{\hat{\A_2}}$, it holds that $w\in\lang{\hat\A_1}$. Therefore, $\lang{\hat\A_1} = \lang{\hat \A_2}$.
The second direction is trivial.
\end{proof}

Every NFA has an equivalent DFA, and DFAs have a canonical form, which is a minimal DFA. We use this property to define a canonical form for $\nfhf$ as minimal deterministic sequence-complete $\nfhf$ with a minimal number of variables. 


\subsubsection{A canonical form for $\nfhe$}

We say that an $\nfhf$ $\A'$ is {\em permutation complete} if for every word $w$, it holds that $\hat \A'$ accepts $w$ iff it accepts every permutation of $w$. 

\begin{lemma}
Let $\A$ be an $\nfhe$.
Let $\A'$ be a permutation-complete $\nfhe$ over $\Sigma$ and $X$ with the following property: for every word $w$, the underlying NFA $\hat\A$ accepts a word $w$ iff $\hat\A'$ accepts all permutations of $w$. %Notice that this is the dual property of the one listed for $\nfhf$.
Then $\lang{\A} =\lang{\A'}$.
\end{lemma}
\begin{proof}
For the first direction, since $\lang{\hat\A}\subseteq \lang{\hat\A'}$, we have $\lang{\A}\subseteq \lang{\A'}$.
For the second direction, let 
$S\in\lang{A'}$. Then there exists $v:S\rightarrow X$ such that $\zip(v)\in\lang{\hat\A'}$. Then
$\zip(v)$ is a permutation of some word $\zip(v')\in\lang{\hat\A}$.  
According to the semantics of the $\exists$ quantifier, we have that $\A$ accepts $S$. 
\end{proof}

\begin{lemma}
Every $\nfhe$ has an equivalent permutation-complete $\nfhe$ over the same set of variables. 
\end{lemma}
\begin{proof}
Similarly to the case of $\nfhf$, we construct $\A'$ given $\A$ by constructing $A_\zeta$ for every permutation $\zeta$ of $(1,2,\ldots k)$.
In this case, the $\nfhe$ $\A'$ is then obtained from $\A$ by replacing the underlying NFA with $\bigcup_{\zeta\in\Gamma} A_\zeta$, where $\Gamma$ is the set of permutations of $(1,2,\ldots k)$.
\end{proof}

\begin{lemma}
Let $\A_1$ and $\A_2$ be two permutation-complete $\nfhe$ over the same set of variables $X$.
Then $\lang{\A_1}=\lang{\A_2}$ iff  $\lang{\hat\A_1} = \lang{\hat \A_2}$.
\end{lemma}
\begin{proof}
For the first direction, let $w\in\lang{\hat\A_1}$. Then $\unzip{w}\in\lang{\A_1}$. 
Then, by the semantics of the $\exists$ quantifier, there exists some permutation $w'$ of $w$ such that $w'\in\lang{\hat\A_2}$.
Since $\A_2$ is permutation-complete, we have that $w\in\lang{\hat\A_2}$. A similar argument can be made to show that for every $w\in\lang{\hat{\A_2}}$, it holds that $w\in\lang{\hat\A_1}$. Therefore, $\lang{\hat\A_1} = \lang{\hat \A_2}$.
The second direction is trivial.
\end{proof}

We define a canonical form for $\nfhe$ as a minimal deterministic permutation-complete $\nfhe$ with a minimal number of variables.




\stam{

\subsubsection{A canonical form for $\nfhef$}

The canonical form for $\nfhef$ is a combination of the canonical forms for $\nfhf$ and $\nfhe$.

We denote an $\nfhef$ $\A$ with a quantification condition $\alpha = \exists x_1\ldots \exists x_t\forall k_{t+1}\ldots \forall x_k$ as a $(k,t)-\nfhef$. 

We say that a $(k,t)-\nfhef$ $\A$ is {\em permutation-sequence complete} if $\A$ accepts a word $w$ iff $\A$ accepts every sequence $w'$ of $w$ in which $(w'[1],\ldots w'[t])$ is a permutation of $(w[1],\ldots w[t])$. 

\begin{lemma}\label{lem:nfhef1}
Let $\A$ be a $(k,t)-\nfhef$. 
Let $\A'$ be a $(k-t)-\nfhef$, with the following property.
$w'\in\lang{\A'}$ iff $w'$ is a sequence of a word $w\in\lang{\hat\A}$ such that
\begin{enumerate}
\item
$(w'[1],\ldots w'[t])$ is a permutation of $(w[1],\ldots w[t])$. 
\item
$w_\zeta\in\lang{\hat\A}$ for every sequence $\zeta$ such that $\zeta = (1,2,\ldots t,i_{t+1},\ldots i_k)$. That is, $\hat\A$ accepts all sequences of $w$ that agree with $w$ on its first $t$ words. 
\end{enumerate}
Then $\lang{\A} = \lang{A'}$.
 \end{lemma}
\begin{proof}
For the first direction, let $S\in\lang{A}$. According to the semantics of the quantifiers, there exist $w_1,\ldots w_t\in S$, such that for every $w_{t+1},\ldots w_{k}$, it holds that $\hat\A$ accepts $\zip(w_1,\ldots w_k)$.
Then, by the semantics of the $\forall$ quantifier, $\hat\A$ accepts all words of the form $\zip(w_1,w_2,\ldots w_t,w'_{t+1},\ldots w'_k)$, where $w'_{t+1},\ldots w'_k$ are taken from $\{w_1,\ldots w_k\}$. Then, by the conditions described above, we have that $\hat\A'$ accepts $\zip(w_1,\ldots w_k)$, and therefore $S\in\lang{A'}$.

For the second direction, let $S\in\lang{A'}$. 
Then there exist $w_1,\ldots w_t\in S$, such that for every $w_{t+1},\ldots w_{k}$, it holds that $\hat\A'$ accepts $w' = \zip(w_1,\ldots w_k)$.
Then, by the conditions described above, $w'$ is a sequence of a word $w\in\lang{A}$ such that $\{w[1],\ldots w[t]\} = \{w'[1], \ldots w'[t]\} = \{w_1,\ldots w_t\}$ (though these words do not necessarily appear in the same order in $w,w'$), and such that $w[i] = w'[i]$ for every $t<i\leq k$. 
Then, by the semantics of the quantifiers, we have that $S\in\lang{\A}$.
\end{proof}

Notice that the $\nfhef$ $\A'$ of Lemma~\ref{lem:nfhef1} is permutation-sequence complete.

\begin{lemma}
Every $(k,t)-\nfhef$ $\A$ has an equivalent permutation-sequence complete $(k,t)-\nfhef$ $\A'$.
\end{lemma}
\begin{proof}
We construct $\A'$ in two steps. 
First, we construct $\A''$ that is the intersection of every $\A_\zeta$ for a sequence $\zeta$ of the form $(1,2,\ldots t, i_{t+1},\ldots i_k)$, as for $\nfhf$ over the last $k-t$ indices, and then we construct $\A'$ as the union of $\A''_\xi$ for every permutation $\xi$ of the form $(i_1,\ldots i_t,t+1,\ldots k)$, as for $\nfhe$ over the first $t$ indices.
\end{proof}

\begin{lemma}
Let $\A_1$ and $\A_2$ be two permutation-sequence complete $(k,t)-\nfhef$.
Then $\lang{\A_1}=\lang{\A_2}$ iff  $\lang{\hat\A_1} = \lang{\hat \A_2}$.
\end{lemma}
\begin{proof}
Suppose that $\lang{\A_1}=\lang{\A_2}$. Let $w\in\lang{\hat\A_1}$. 
Since $\A_1$ is permutation-sequence complete, we have that $w_\zeta\in\lang{\A_1}$ for every sequence $\zeta$ of the form $(1,\ldots t,i_{t+1},\ldots i_k)$. By the semantics of the quantifiers, we have that $\unzip{w}\in\lang{A_1}$.
Therefore, $\unzip{w}\in\lang{A_2}$, and so, according to the semantics of the quantifiers and the order of words in $w$, 
there exists a word $w'\in\lang{\hat\A_2}$ such that $(w'[1],\ldots w'[t])$ is a permutation of $(w[1],\ldots w[t])$, and $(w'[t+1],\ldots w'[k])$ is a permutation of $(w[t+1],\ldots w[k])$. Since $\A_2$ is permutation-sequence complete, we have that $w\in\lang{\hat\A_2}$. A similar argument shows that $\lang{\hat\A_2}\subseteq \lang{\hat\A_1}$.
The second direction is trivial.
\end{proof}

\begin{lemma}
Let $\A$ be a $(k,t)-\nfhef$, such that $k$ is the minimal number of variables needed for expressing $\lang{\A}$.
Then $\hat\A$ accepts a word $w$ in which $w[1],\ldots w[t]$ are all distinct. 
\end{lemma}
\begin{proof}
Suppose by way of contradiction that for every $w\in\lang{\hat\A}$, two of the first $t$ words of in $w$ are equal. Due to the semantics of the $\exists$ quantifier, we can also 
\end{proof}


\begin{lemma}
Let $\A_1$ and $\A_2$ be permutation-sequence complete $\nfhef$ for the same hyperlanguage $\cal L$, both with a minimal set of variables. Then the quantification conditions of $\A_1$ and $\A_2$ are equal.
\end{lemma}
\begin{proof}
Suppose that $\A_1$ is a $(t,k)-\nfhef$, and $\A_2$ is a $(t,m)-\nfhef$.
Suppose by way of contradiction, and without loss of generality, that $t < m$. 
Let $S\in\lang{\A_1}$. 
\end{proof}

We define a canonical form for $\nfhef$ as a minimal deterministic permutation-sequence complete $\nfhef$ with a minimal number of variables. 


}% of stam









